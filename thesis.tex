\documentclass[12pt,a4paper,oneside,ngerman]{report}

\usepackage[a4paper,left=4.5cm,right=2.5cm,top=2.5cm,bottom=2cm]{geometry}
\usepackage{fancyhdr} % Kopf- und Fußzeile
\usepackage[onehalfspacing]{setspace} % 1,5 facher Zeilenabstand
\usepackage{graphicx} % Bilder einbinden
\usepackage{titling} % Referenzen auf Titel / Autor etc. im Text
\usepackage[hidelinks]{hyperref} % Hyperlinks im Inhaltsverzeichnis
\usepackage[german]{babel} % Übersetzungen
\usepackage{acronym} % Abkürzungsverzeichnis
\usepackage[backend=biber,style=LNI]{biblatex} % Literaturverzeichnis
\usepackage{csquotes} % Babel und Biblatex Sprachanpassungen
\usepackage{framed} % Kästen für den Sperrvermerk
\addbibresource{literature.bib}

% V A R I A B L E N
%---------------------------------------------
\author{-Hier Autor eintragen/-}
\title{-Hier Titel eintragen/-}
\newcommand{\degreeCourse}{-Hier Studiengang eintragen/-}
\newcommand{\studentNumber}{-Hier Matrikelnummer eintragen/-}
\newcommand{\company}{-Hier Unternehmen eintragen/-}
\newcommand{\filingDate}{-Hier Abgabedatum eintragen/-}
\newcommand{\firstExaminer}{-Hier Erstprüfer eintragen/-}
\newcommand{\secondExaminer}{-Hier Zweitprüfer eintragen/-}
\providetoggle{blockingNotice}
\settoggle{blockingNotice}{true} % Sperrvermerk anzeigen oder ausblenden

\setlength{\headheight}{14.5pt} % Damit fancyhdr keine Fehlermeldung generiert
\setlength{\emergencystretch}{1.5em} % Für passende Wortumbrüche

% Header und Footer Definitionen
%---------------------------------------------
\fancypagestyle{onlyPageNumbers}{ % Zeigt nur Footer mit Seitenzahl rechts
    \fancyhf{}
    \fancyfoot[C]{\thepage}
    \fancypagestyle{plain}{%
        \fancyhf{}%
        \fancyfoot[R]{\thepage}%
        \renewcommand{\headrulewidth}{0pt}
    }
}

\fancypagestyle{toc}{ % Zeigt Footer + Header auf zweiter Seite (ToC)
    \fancyhf{}
    \fancyfoot[C]{\thepage}
    \fancypagestyle{plain}{%
        \fancyhf{}%
        \fancyhead[R]{\nouppercase{\leftmark}}
        \fancyfoot[R]{\thepage}%
    }
}

\fancypagestyle{runningTitleAndPageNumbers}{ % Zeigt Header + Footer
    \fancyhf{}
    \fancyhead[R]{\nouppercase{\leftmark}}
    \fancyfoot[R]{\thepage}
    \fancypagestyle{plain}{%
        \fancyhf{}%
        %\renewcommand{\headrulewidth}{0pt} % Kommentiere entweder diese oder die nächste Zeile aus: Header auf der ersten Seite eines Kapitels anzeigen oder nicht
        \fancyhead[R]{\nouppercase{\leftmark}}
        \fancyfoot[R]{\thepage}%
    }
}

% Standardformate überschreiben
%---------------------------------------------
\makeatletter
    % Format der Chapterüberschriften überschreiben
    \renewcommand*\@makechapterhead[1]{
        {
            \parindent\z@\raggedright\normalfont
            \Huge\bfseries
            \thechapter\hspace{12px} #1\par
            \nobreak\vskip 10\p@
        }
    }
    \renewcommand*\@makeschapterhead[1]{
        {
            \parindent\z@\raggedright\normalfont
            \Huge\bfseries
            #1
            \nobreak\vskip 10\p@
        }
    }

    % Margin bei Abbildungs- und Tabellenverzeichnis entfernen
    \renewcommand*\l@figure{\@dottedtocline{1}{0em}{2.3em}} % Default: 1.5em/2.3em
    \let\l@table\l@figure
\makeatother

\newcommand*{\fullref}[1]{\hyperref[{#1}]{\ref*{#1} \nameref*{#1}}} % Referenzen auf Kapitel und Name mit \fullref

% D O K U M E N T E N A N F A N G
%---------------------------------------------
\pagestyle{onlyPageNumbers}
\begin{document}

% D E C K B L A T T
%---------------------------------------------
\newgeometry{margin=3cm} % Seitenränder für Deckblatt anders

\begin{titlepage}
    \begin{flushleft}
        \includegraphics[scale=1.25]{res/fh_logo.pdf}
    \end{flushleft}

    \vspace*{\stretch{1.0}}
    \begin{center}
        \Large\textbf\thetitle\\
        \vspace{1em}
        \large{Abschlussarbeit im Studiengang {\degreeCourse}}\\
        {am Fachbereich Wirtschaft der FH Münster}

    \end{center}
    \vspace*{\stretch{2.0}}

    \begin{flushleft}
        \begin{tabular}{l l}
            {Vorgelegt von:} & \theauthor \\
            {Matrikelnummer:} & \studentNumber \\
            {Erstprüfer:} & {\firstExaminer} \\
            {Zweitprüfer:} & {\secondExaminer} \\
            {Unternehmen:} & {\company} \\
            {Abgabedatum:} & {\filingDate} \\
        \end{tabular}
    \end{flushleft}
\end{titlepage}

\restoregeometry % Seitenränder auf Standard setzen

% A B S T R A C T
%---------------------------------------------
\chapter*{Abstract}
\pagenumbering{Roman}

% \input{helpers/loremipsum}

% S P E R R V E R M E R K
%---------------------------------------------
\iftoggle{blockingNotice}{
    \chapter*{Sperrvermerk}

    \noindent{Die vorliegende Bachelorarbeit mit dem Titel:}

    \begin{framed}\noindent\thetitle\end{framed}

    \noindent{beinhaltet interne und vertrauliche Informationen der Firma:}

    \begin{framed}
        \noindent\company{ .}
    \end{framed}

    \noindent{Die Weitergabe des Inhalts der Arbeit und eventuell beiliegender
    Zeichnungen und Daten, im Gesamten oder in Teilen, ist grundsätzlich
    untersagt. Es dürfen keinerlei Kopien oder Abschriften – auch in
    digitaler Form – gefertigt werden. Ausnahmen bedürfen der
    schriftlichen Genehmigung der o.g. Firma.}

    \vspace{1.5cm}\begin{flushleft}
    \begin{tabular}{lp{2em}l}
        \hspace{5cm}   && \hspace{5cm} \\\cline{1-1}\cline{3-3}
        Ort, Datum     && Unterschrift
    \end{tabular}
\end{flushleft}
}

% I N H A L T S V E R Z E I C H N I S
%---------------------------------------------
\tableofcontents{\protect\thispagestyle{plain}}
%\pagestyle{toc} % Damit auf der zweiten Seite im ToC ein Header angezeigt wird, auskommentieren wenn ToC mehrseitig

% A B K Ü R Z U N G S V E R Z E I C H N I S
%---------------------------------------------
\chapter*{Abkürzungsverzeichnis}
\pagestyle{onlyPageNumbers}
\vspace{0.8em}
\addcontentsline{toc}{chapter}{Abkürzungsverzeichnis}

\begin{acronym}[<Länge Abkürzung>...] % Abstand zwischen Abkürzung und Erklärung durch Angabe der längsten Abkürzung + 3 Zeichen
    \acro{ABC}[ABC]{Aaaa Beee Ceee}\\
\end{acronym}

% A B B I L D U N G S V E R Z E I C H N I S
%---------------------------------------------
\listoffigures
\addcontentsline{toc}{chapter}{Abbildungsverzeichnis}

% T A B E L L E N V E R Z E I C H N I S
%---------------------------------------------
\listoftables
\addcontentsline{toc}{chapter}{Tabellenverzeichnis}

% T E X T T E I L
%---------------------------------------------

% KAPITEL 1
\chapter{Kapitel 1}
\pagenumbering{arabic}
\pagestyle{runningTitleAndPageNumbers}

\input{helpers/loremipsum}

\section{Abschnitt 1}

\input{helpers/loremipsum}

\subsection{Unterabschnitt 1}

\input{helpers/loremipsum}

\subsection{Unterabschnitt 2}

\input{helpers/loremipsum}

\section{Abschnitt 2}

\input{helpers/loremipsum}

% KAPITEL 2
\chapter{Kapitel 2}

\input{helpers/loremipsum}

% L I T E R A T U R V E R Z E I C H N I S
%---------------------------------------------
\begingroup
    \setlength{\emergencystretch}{3em} % Damit Einträge umgebrochen werden
    \printbibliography
\endgroup

\addcontentsline{toc}{chapter}{Literaturverzeichnis}

% A N H A N G
%---------------------------------------------
\newcommand{\attachment}{Anhang}
\chapter*{\attachment}
\addcontentsline{toc}{chapter}{\attachment}
\markboth{\attachment}{}
\setcounter{section}{0} % Fange bei den Sections wieder neu an zu zählen
\renewcommand{\thesection}{\Alph{section}} % Nummerierung von Section formatieren 
\addtocontents{toc}{\setcounter{tocdepth}{0}} % Zeige Chapter und Sections nicht im Inhaltsverzeichnis

\section{Erster Anhang}

\input{helpers/loremipsum}

\section{Zweiter Anhang}

\input{helpers/loremipsum}

\addtocontents{toc}{\setcounter{tocdepth}{2}} % Zeige Chapter und Sections wieder normal im Inhaltsverzeichnis

% E I D E S S T A T T S E R K L Ä R U N G
%---------------------------------------------
\chapter*{Eidesstattliche Erklärung}
\thispagestyle{empty} % Keine Seitennnummerierung und Kopf-/Fußzeile

Ich versichere, dass ich diese schriftliche Arbeit selbstständig
angefertigt, alle Hilfen und Hilfsmittel angegeben und alle wörtlich
oder im Sinne nach aus Veröffentlichungen oder anderen Quellen,
insbesondere dem Internet entnommene Inhalte, kenntlich gemacht habe.

\vspace{1.5cm}\begin{flushleft}
    \begin{tabular}{lp{2em}l}
        \hspace{5cm}   && \hspace{5cm} \\\cline{1-1}\cline{3-3}
        Ort, Datum     && Unterschrift
    \end{tabular}
\end{flushleft}

\end{document}